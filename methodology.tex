\chapter{Methodology}

This research is geared towards finding a correlation between users' biometric data to their performance in a first-person shooter gaming scenario. The quality of the user data 
collected was critical in ensuring that the research aims are achieved. For Quality Assurance purposes, a systematic step-by-step approach has been designed for the data-capturing 
process. An activity monitoring device was distributed to the volunteers in the form of a Polar Watch. The individuals were instructed to use these devices at pre-designated times 
prior to undertaking a test in the form of a First-Person Shooter Game. Results from the test, was than paired with their biometric data for further analysis.

\section{Data Collection}

For the purpose of the research, different categories of data was collected from the volunteers through the monitoring device to help achieve the stated goals. As personal data was
being collected, an application to the Ethicas Research and Ethics Committee was made to ensure that the data collection process was in compliance with the General Data Protection. 
The Taught Programme Research Ethics Approval Application Form can be found on appendix \ref{appendix}. It was an extensive and iterative process with the Ethicas Research and Ethics 
office to ensure that the data collection process was in compliance with the General Data Protection Regulation (GDPR). Unfortunately, it took longer than expected to get the approval
from the Ethicas Research and Ethics Committee, which lead to an unexpected delay in the data collection process. The data collection process was divided into two main categories:

\begin{itemize}
    \item Biometric Data
    \item Test Game
\end{itemize}




These data relate 
to the volunteer’s physical information, sleep data, daily activities, and nightly recharge.
For the Test Game, were instructed to play a total of three different categories of test to gauge their performance in these tests. The tests are Audio Test, Visual Test, and Fine Motor
Test. The tests were designed to be played sequentially and typically take a total of 15 minutes to conclude.

\subsection{Data Collection Procedure}

The data collection process followed the below-listed steps to maximize throughput and minimize the average time spent processing each participant.

\textbf{Location:}

The data collection effort was carried out at the Gym on the Atlantic Technological University (ATU) Galway campus.

\textbf{Time:}

The following weekly schedule was available for the volunteers at various times that best suit their personal schedules.

· Tuesdays: 12:00 - 13:00, 15:00 - 16:00
· Wednesdays: 12:00 - 13:00, 15:00 - 16:00
· Thursdays: 12:00 - 13:00, 15:00 - 16:00
· Fridays: 13:00 - 15:00

\textbf{Biometric Data Procedure:}



\chapter{Methodology}

This research is geared towards finding a correlation between users' biometric data to their performance in a first-person shooter gaming scenario. The quality of the user data 
collected was critical in ensuring that the research aims are achieved. For Quality Assurance purposes, a systematic step-by-step approach has been designed for the data-capturing 
process. An activity monitoring device was distributed to the volunteers in the form of a Polar Watch. The individuals were instructed to use these devices at pre-designated times 
prior to undertaking a test in the form of a First-Person Shooter Game. Results from the test, was than paired with their biometric data for further analysis.

\section{Recruitment of Volunteers}
??



\section{Data Collection}

For the purpose of the research, different categories of data was collected from the volunteers through the monitoring device to help achieve the stated goals. As personal data was
being collected, an application to the Ethicas Research and Ethics Committee was made to ensure that the data collection process was in compliance with the General Data Protection.

\par 
The Taught Programme Research Ethics Approval Application Form can be found on appendix \ref{appendix}. It was an extensive and iterative process with the Ethicas Research and Ethics 
office to ensure that the data collection process was in compliance with the General Data Protection Regulation (GDPR). Unfortunately, it took longer than expected to get the approval
from the Ethicas Research and Ethics Committee, which lead to an unexpected delay in the data collection process. The data collection process was divided into two main categories:

\subsubsection*{Biometric Data}

For the very first time processing a volunteer, some user information was needed to have them registered with the activity monitoring device and their data was saved to the manufacturer'
repository. The Activity Monitoring Device used on this research was the Polar Vantage Smart Watch. The watch is capable of monitoring and capturing user’s biometric data and saving the
data to a repository where it can be accessed online via an API. The device is commercially available and accessible to the public. The rationale for using the Polar Vantage Watch is
because it has been widely used in both academic and industrial research projects and validation. Most importantly, these devices were available to us in such quantity that could satisfy
our research needs.

\subsubsection*{Physical Information}
The following physical information was collected from the volunteers, and stored with the manufacturer (Polar) which is GDPR compliant.:
\begin{itemize}
    \item Sleep Data: According to the manufacturer's manual \cite{polarManual}, the watch is capable of recording the quality and quantity of sleep and also showing how long spent on each stage of sleep. 
    Light Sleep, Deep Sleep, and REM Sleep with the duration of each type of sleep. For the purpose of the research, an aggregated total of the various categories of sleep will be recorded and used.
    Volunteers were instructed to wear the watch to sleep on the previous night before their scheduled test. Unit of measure used for the sleep data is minutes. 
    \item Daily Activities: According to the Activity monitor manufacturers' manual \cite{polarManual}, "Polar device uses an internal 3D accelerometer to record your wrist movements. It analyses the frequency, 
    intensity and regularity of your movements together with your physical information." calories, active calories, active steps and their respective durations are collected from through the device. For the 
    purpose of the research, the active steps was used. Volunteers were expected to wear the device on the previous day for this data to be available. Unit of measure used for the daily activities is count of steps.
    \item Nightly Recharge: From the Activity monitor manufacturers' manual \cite{polarManual}, the Nightly Recharge is recorded as follows: " is an overnight recovery measurement that shows how well your body has
    coped with overall stress you have experienced lately." The parameters, measured during roughly the first four hours of your sleep are heart rate, heart rate variability and breathing rate. For the purpose 
    of the research, the following parameters and derived parameters will be used:
    \begin{itemize}
        \item Heart Rate Average (bpm)
        \item Heart Rate Maximum (bpm)
    \end{itemize}
\end{itemize}


These data relate 
to the volunteer’s physical information, sleep data, daily activities, and nightly recharge.
For the Test Game, were instructed to play a total of three different categories of test to gauge their performance in these tests. The tests are Audio Test, Visual Test, and Fine Motor
Test. The tests were designed to be played sequentially and typically take a total of 15 minutes to conclude.

\subsection{Data Collection Procedure}

The data collection process followed the below-listed steps to maximize throughput and minimize the average time spent processing each participant.

\textbf{Location:}

The data collection effort was carried out at the Gym on the Atlantic Technological University (ATU) Galway campus.

\textbf{Time:}

The following weekly schedule was available for the volunteers at various times that best suit their personal schedules.

· Tuesdays: 12:00 - 13:00, 15:00 - 16:00
· Wednesdays: 12:00 - 13:00, 15:00 - 16:00
· Thursdays: 12:00 - 13:00, 15:00 - 16:00
· Fridays: 13:00 - 15:00

\textbf{Biometric Data Procedure:}


Test Game
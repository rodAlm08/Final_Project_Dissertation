\chapter{Machine Learning Models}

\section{Introduction}
As explained in dept in the methodology chapter, the biometric data collected through the watches generated a total of 6 independent variables:

\begin{itemize}
    \item Heart Rate Maximum \textit{bm\_HR\_max}
    \item Heart Rate Average \textit{bm\_HR\_avg}
    \item Heart Rate Variability \textit{bm\_HR\_var}
    \item Activity Steps \textit{bm\_act\_steps}
    \item Sleep \textit{bm\_sleep}
\end{itemize}

The dependent variables that were collected through the gaming tests are:

\begin{itemize}
    \item Fine Motor Tracking Time \textit{fm\_avg\_trk\_time}
    \item Fine Motor Accuracy \textit{fm\_accuracy}
    \item Visual Average Response Time \textit{vx\_avg\_res\_time}
    \item Visual Shot Accuracy \textit{vx\_shot\_accuracy}
    \item Visual Target Accuracy \textit{vx\_trg\_accuracy}
    \item Audio Average Response Time \textit{au\_avg\_res\_time}
\end{itemize}

The goal based on the research question was to find the correlation between the independent and dependent variables.

\section{Data Pre-processing}
To have the dataset suitable for the machine learning models, the data was pre-processed. It was found that were many null values that needed to be handled, 
it was removed from the dataset as it was not possible to impute the values. The \section{Activity Steps} and \section{Heart Rate Maximum} were found to have
outliers that were removed from the dataset. The data was then scaled using StandardScaler from the sklearn library. This was done to ensure that all the features
contribute equally to the result.
The columns \textit{\_id}, \textit{\_date} and \textit{\_user} were removed from the dataset as they were not needed for the machine learning models.






\section{Feature Engineering}





\section{Classical ML Regression Model}








\section{Neural Network Regression Model}




\chapter{Machine Learning Models}

\section{Introduction}
As explained in dept in the methodology chapter, the biometric data collected through the watches generated a total of 6 independent variables:

\begin{itemize}
    \item Heart Rate Maximum \textit{bm\_HR\_max}
    \item Heart Rate Average \textit{bm\_HR\_avg}
    \item Heart Rate Variability \textit{bm\_HR\_var}
    \item Activity Steps \textit{bm\_act\_steps}
    \item Sleep \textit{bm\_sleep}
\end{itemize}

The dependent variables that were collected through the gaming tests are:

\begin{itemize}
    \item Fine Motor Tracking Time \textit{fm\_avg\_trk\_time}
    \item Fine Motor Accuracy \textit{fm\_accuracy}
    \item Visual Average Response Time \textit{vx\_avg\_res\_time}
    \item Visual Shot Accuracy \textit{vx\_shot\_accuracy}
    \item Visual Target Accuracy \textit{vx\_trg\_accuracy}
    \item Audio Average Response Time \textit{au\_avg\_res\_time}
\end{itemize}

The goal based on the research question was to find the correlation between the independent and dependent variables. When approaching a machine learning problem, one of the fundamental
considerations is whether the problem is a regression or classification problem. 

\subsection*{Classification vs Regression Problem}
The main difference between a classification and regression problem is the type of dependent variable. The dependent variable is the variable that are being predicted. In 
classification, the dependent variable is categorical, meaning it can take one of a limited number of values. Examples includes predicting whether an email is spam or not,
predicting whether a patient has a disease or not. In regression, the dependent variable is continuous and numerical, meaning it can take any value within a range. Examples 
includes predicting house prices, stock prices, temperature. In this project, the dependent variables are continuous and numerical, making it a regression problem. The goal 
was to predict the dependent variables based on the biometric data collected from the watches. 

\section{Data Exploration}
The first step in the machine learning process was to explore the dataset. The dataset was loaded into a pandas dataframe and the first 5 rows were displayed to get an overview of the
data. The shape of the dataset was checked to see the number of rows and columns. The data types of the columns were checked to ensure that the data types were correct. The summary
statistics of the dataset were checked to see the mean, median, standard deviation, minimum and maximum values of the dataset. The correlation between the independent and dependent
variables were checked to see if there was any correlation between the variables. The correlation was visualized using a heatmap \ref{fig:correlation_heatmap} to see the correlation 
between the variables.

\begin{figure}[H]
    \centering
    \includegraphics[width=1\textwidth]{images/correlation.png}
    \caption{Correlation Heatmap}
    \label{fig:correlation_heatmap}
\end{figure}

The correlation heatmap shows the correlation between the independent and dependent variables. The correlation values range from -1 to 1. A value of 1 indicates 
a strong positive correlation, a value of -1 indicates a strong negative correlation and a value of 0 indicates no correlation. The correlation heatmap shows 
the following correlations between the dependent and the independent variables:

\subsubsection*{Heart Rate Average (\textit{bm\_HR\_avg})}
\begin{itemize}
    \item Fine Motor Tracking Time (\textit{fm\_avg\_trk\_time}): A negative correlation of \\ -0.41 suggests that as the average heart rate increases, the fine motor
    tracking time tasks decreases, indicating that individuals with faster heart rates tend to complete fine motor tasks a bit quicker.
    \item Fine Motor Accuracy (\textit{fm\_accuracy}): A positive correlation of 0.16 indicates a weak relationship suggesting that a higher heart rate average might be very 
    slightly associated with higher fine motor accuracy. It shows that there is a small tendency for individuals with a higher heart rate to be slightly more precise
    in tasks that need fine motor skills.
    \item Visual Average Response Time (\textit{vx\_avg\_res\_time}): A negative correlation of -0.35 suggests a moderate relationship where a higher average heart rate
    is associated with faster response times. It indicates that individuals with higher heart rates also tend to react faster to visual things.
    \item Visual Shot Accuracy (\textit{vx\_shot\_accuracy}): A positive correlation of 0.38 indicates a moderate relationship, suggesting that individuals with higher heart
    rate might have a better accuracy in shooting tasks.
    \item Visual Target Accuracy (\textit{vx\_trg\_accuracy}): A positive correlation of 0.22 indicates a weak relationship, suggesting that higher heart rates averages might
    be associated with slightly better visual target accuracy.
    \item Audio Average Response Time (\textit{au\_avg\_res\_time}): A very weak negative correlation of -0.05 suggest almost no relationship between average heart rate
    and audio response time. It indicates there is no real connection between heart rate and how quickly the individuals responds to sounds.
\end{itemize}

\subsubsection*{Hear Rate Variability (\textit{bm\_HR\_var})}

\begin{itemize}
    \item Fine Motor Tracking Time (\textit{fm\_avg\_trk\_time}): A positive correlation of 0.20 suggests a weak association where greater heart rate variability 
    might be associated with slightly longer fine motor tracking time tasks.
    
    \item Fine Motor Accuracy (\textit{fm\_accuracy}): A negative correlation of -0.14 suggest a very weak inverse relationship, where higher heart rate variability
    could be slightly associated with a decrease in fine motor accuracy.
    
    \item Visual Average Response Time (\textit{vx\_avg\_res\_time}): A positive correlation of 0.13 indicates very weak relationship, with slightly tendency for
    higher heart rate variability to be associated with longer visual response times. 
    
    \item Visual Shot Accuracy (\textit{vx\_shot\_accuracy}): A negative correlation of -0.27 indicates a weak to moderate inverse relationship, suggesting that
    higher rate variability might be associated with a decrease in visual shot accuracy. It suggests that individuals with higher heart rate variability might 
    have a lower accuracy in shooting tasks.
    
    \item Visual Target Accuracy (\textit{vx\_trg\_accuracy}): A negative correlation of -0.14 indicates a weak inverse relationship, suggesting that higher heart 
    variability could slightly correlate with lower visual target accuracy. It suggests that those with more variation in their heart rate might not be quite as good 
    at tasks that involve quickly identifying and selecting targets.
    
    \item Audio Average Response Time (\textit{au\_avg\_res\_time}): A positive correlation of 0.22 suggests a weak relationship, indicating that higher heart rate
    variability might be associated with slightly longer audio response times.
    
\end{itemize}


\subsubsection*{Sleep (\textit{bm\_sleep})}

\begin{itemize}
    \item Fine Motor Tracking Time (\textit{fm\_avg\_trk\_time}): A positive correlation of 0.10 suggests a very weak relationship, indicating that longer sleep duration
    may be very slightly correlated with longer fine motor tracking time. 
        
    \item Fine Motor Accuracy (\textit{fm\_accuracy}): With a value of -0.09, this negative correlation is very weak, indicating minimal inverse relationship between sleep
    and fine motor accuracy.
    
    \item Visual Average Response Time (\textit{vx\_avg\_res\_time}): At 0.10, the correlation is weak, suggesting a minimal tendency for more sleep to be associated with slightly
    longer response times in visual tasks.
    
    \item Visual Shot Accuracy (\textit{vx\_shot\_accuracy}): The value of -0.18 indicates a weak negative correlation, suggesting that increased sleep could be potentially be associated
    with a slight improvements in visual shot accuracy. 

    \item Visual Target Accuracy (\textit{vx\_trg\_accuracy}): At -0.04, the correlation is weak, indicating there is almost no relationship between sleep and visual target accuracy.
    
    \item Audio Average Response Time (\textit{au\_avg\_res\_time}): The correlation value of 0.06, which is very weak and suggests there is little to no meaningful correlation between
    sleep and response time to audio stimulus.
    
\end{itemize}


\section{Data Pre-processing}
To have the dataset suitable for the machine learning models, the data was pre-processed. It was found that were many null values that needed to be handled, 
it was removed from the dataset as it was not possible to impute the values. The \textit{Activity Steps} and \textit{Heart Rate Maximum} were found to have
outliers that were removed from the dataset. The data was then scaled using StandardScaler from the sklearn library. This was done to ensure that all the features
contribute equally to the result. The columns \textit{\_id}, \textit{\_date} and \textit{\_user} were removed from the dataset as they were not needed for the machine learning models.
The next step was to split the dataset into training and testing datasets. The training dataset was used to train the machine learning models and the testing dataset was used to evaluate
the model. The dataset was split into 80\% training and 20\% testing. The dependent and independent variables were then separated into X and y. The data was then ready to be used for the machine learning models.


\section{Model Selection and Evaluation}
Various machine learning models were explored to address the research question and predict the dependent variables based on the collected biometric and gaming test. Each team member
focused on developing and evaluating a model to achieve the best predictive performance. 

\subsection{Classical ML Regression Models}
The classical machine learning regression models were explored to predict the dependent variables based on the independent variables. The model were trained using the training dataset 
and evaluated using the testing dataset. The mean squared error was used to evaluate the model. The model with the lowest mean squared error was selected as the best performing model and 
used for further analysis. The following algorithm were explored:

\begin{itemize}
    \item Linear Regression: Chose as a baseline for its simplicity and Interpretability, assuming a linear relationship between the dependent and independent variables.
    \item Random Forest Regressor: Chose for its ability to handle non-linear relationships and its robustness to overfitting. 
    \item Support Vector Regressor: Selected for its ability to handle high-dimensional data and capability to find complex patterns by mapping input data into a higher-dimensional feature space.
    \item K-Nearest Neighbours Regressor: Chose for its simplicity and effectiveness in regression tasks, making predictions based on the average of the k-nearest neighbours in the feature space.
\end{itemize}

Once the best performing model for each dependent variable was selected, the model was then evaluated using the testing dataset and actual versus predicted values were plotted to see
how well the model performed. 

\subsection{Actual vs Predicted Plot}
The actual versus predicted plot was used to visualize how well the model performed. The actual values were plotted against the predicted values to see how well the model predicted the
dependent variables. The plot was used to see if the model was underfitting or overfitting the data. If the points were close to the line, it indicated that the model was performing well.
If the points were scattered, it indicated that the model was not performing well. The plot was used to see if the model was capturing the underlying patterns in the data. Below are the 
actual versus predicted plots for the best performing model for each dependent variable.

\subsection*{Fine Motor Tracking Time}

\begin{figure}[htbp]
    \centering
    \begin{subfigure}[b]{0.49\textwidth}
        \centering
        \includegraphics[width=\textwidth]{images/charts/test_data_fine_motor_tracking_time.png}
        \caption{Actual vs Predicted (Test Data)}
        \label{fig:actual_vs_predicted_fm_avg_trk_time_test}
    \end{subfigure}\hfill
    \begin{subfigure}[b]{0.49\textwidth}
        \centering
        \includegraphics[width=\textwidth]{images/charts/all_data_fine_motor_tracking_time.png}
        \caption{Actual vs Predicted (All Data)}
        \label{fig:actual_vs_predicted_fm_avg_trk_time_all_data}
    \end{subfigure}
    \caption{Fine Motor Tracking Time Actual vs Predicted}
    \label{fig:fine_motor_tracking_time_comparison}
\end{figure}

\subsubsection*{Test Data Evaluation}

\begin{itemize}
    \item \textbf{Trend}: Figure \ref{fig:actual_vs_predicted_fm_avg_trk_time_test} shows the actual versus predicted values for the Fine Motor Average Tracking Time using the test data. The model that performed 
    the best for this dependent variable was the Random Forest Regressor. The model's predictions generally matched the direction of the actual test data's trend. The peaks and troughs of the
    actual data were also found in the predicted data, with some deviations. 
    \item \textbf{Variance}: There is a noticeable variance between the actual and predicted values, specially at the extremes. For instance, the model appears to underestimate the highest values
    and overestimate the lowest values.
    \item \textbf{Consistency}: The model shows decent consistency when the actual values are around the mean but is less consistent at capturing sudden changes in the actual data, such as sharp spikes ot dips.
\end{itemize}

\subsubsection*{All Data Evaluation}

\begin{itemize}
    \item \textbf{Trend}: Figure \ref{fig:actual_vs_predicted_fm_avg_trk_time_all_data} shows the actual versus predicted values for the Fine Motor Average Tracking Time using all the data
    available. When evaluating all the data, the model predictions closely follow the actual data's trend. This indicates that the model has learned the overall behavior of the dataset quite well.
    Similar to the test data, the model captures the general pattern of movement in the actual values, but might not always match the amplitude of changes.
    \item \textbf{Variance}: The variance between the actual and predicted values over all the data seems to be lower compared to the test data. This suggests that the model has been 
    effectively trained to understand the data as a whole. Some exceptions occur where the actual values show significant deviation from the mean. In these areas, the predictions do not fully 
    capture the extent of the actual values and show some deviation.
    \item \textbf{Consistency}: The model shows good consistency in predicting the values when the actual values are around the mean. The predictions often match the actual values quite closely.
\end{itemize}

Overall the Random Forest Regressor model performed well in both scenarios, capturing the general trends of the Fine Motor Tracking Time and showing good consistency in prediction.
The model, however, seems to struggle with accurately predicting the more extreme values in the dataset. It highlights the need for improving the model's performance on the more complex
or extreme segments of the data.



% Figure \ref{fig:actual_vs_predicted_fm_avg_trk_time} shows the actual versus predicted values for the Fine Motor Average Tracking Time. The model that performed the best for this dependent
% variable was the Random Forest Regressor. 

\subsubsection*{Results and Analysis}
The comparison between the actual and predicted values for Fine Motor Tracking Time shows the following characteristics:

\begin{itemize}
    \item Trend Follow-Up: The model demonstrated a strong ability to follow the overall trend of the actual Fine Motor Tracking Time across the different values. 
    \item Consistency: In many segments, particularly where the data points did not reach extreme highs or lows, the model's predictions closely matched the actual values, indicating good consistency. 
    \item Variance: There is variability in the model's predictions, with some points showing a strong match to the actual data, and others showing a larger deviation.
\end{itemize}

In conclusion, the Random Forest Regressor model serves as a solid baseline for predicting Fine Motor Tracking Time.

\subsubsection*{Fine Motor Accuracy}


% \begin{figure}[H]
%     \centering
%     \includegraphics[width=1\textwidth]{images/output_fm_accuracy_rf.png}
%     \caption{Actual vs Predicted Fine Motor Accuracy}
%     \label{fig:actual_vs_predicted_fm_accuracy}
% \end{figure}

% Figure \ref{fig:actual_vs_predicted_fm_accuracy} shows the actual versus predicted values for the Fine Motor Accuracy. The model that performed the best for this dependent variable was
% the Random Forest Regressor. 

\subsubsection*{Results and Analysis}
The comparison between the actual and predicted values for Fine Motor Accuracy shows the following characteristics:




\section{Neural Network Regression Model}



